\section{Building against libremix}
\label{group__building}\index{Building against libremix@{Building against libremix}}
\subsection{Using GNU autoconf}\label{autoconf}
If you are using GNU autoconf, you do not need to call pkg-config directly. Use the following macro to determine if libremix is available:

\small\begin{alltt}
 PKG\_CHECK\_MODULES(REMIX, remix $>$= 0.2.0,
                   HAVE\_REMIX="yes", HAVE\_REMIX="no")
 if test "x$HAVE\_REMIX" = "xyes" ; then
   AC\_SUBST(REMIX\_CFLAGS)
   AC\_SUBST(REMIX\_LIBS)
 fi
 \end{alltt}\normalsize 


If libremix is found, HAVE\_\-REMIX will be set to \char`\"{}yes\char`\"{}, and the autoconf variables REMIX\_\-CFLAGS and REMIX\_\-LIBS will be set appropriately.\subsection{Determining compiler options with pkg-config}\label{pkg-config}
If you are not using GNU autoconf in your project, you can use the pkg-config tool directly to determine the correct compiler options.

\small\begin{alltt}
 REMIX\_CFLAGS=`pkg-config --cflags remix`\end{alltt}\normalsize 


\small\begin{alltt} REMIX\_LIBS=`pkg-config --libs remix`
 \end{alltt}\normalsize 
 

